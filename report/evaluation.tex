%!TEX root = ./report.tex
\section{Evaluation}
\label{sec:evaluation}
In this section we evaluate the solution, with reference to the requirements of Section \ref{subsec:requirements}. Furthermore, we discuss the methods used to verify the correctness of the product. Recall that a usability evaluation of the Pipes DSL is outside the scope of the project as also described in Section \ref{subsec:requirements}. We conclude this section by evaluating our use of EMF for this particular domain.

% Maybe: Discussion of applicability of product. What should be done in order to make this project usable in a real world scenario?

\subsection{Requirements Evaluation}
\label{subsec:requirements_evaluation}
\paragraph{Working with a Partial Model} By using Xtend classes to define workflow components we are able to modularize the individual steps of the prototype. These steps are chained using MWE2 workflows. This achieves reusability and verifiability of each of the modular components. Note how several workflow components have been reused in both workflows. We presented the informal IFC subset definition that we desired in Section \ref{subsec:problem_analysis}, but probably the clearest depiction of the problem domain was achieved by creating the simple Ecore meta model shown in Figure \ref{fig:pipes_dsl_ecore_model}.

\paragraph{Correct Meta Model} As explained in Section \ref{subsec:ifc_meta_model} the prototype uses the official ifcXML XSD to generate an Ecore meta model that is both correct and easily updatable. However, there were certain serious drawbacks to this solution.

During this project, we encountered several problems with EMF, which we believe are due to the large size of the IFC meta model. When generating the IFC meta model, several bugs were encountered, such as some methods not being generated (FIXME: reference), and methods exceeding the maximum size allowed by the Java Virtual Machine (FIXME: reference). While we were able to work around these bugs after some effort, they indicated that EMF had trouble handling meta models of this size (more than 2,000 elements).

When working with an instance of a realistic model, we experienced load and save times in excess of 10 minutes using the default XML resource deserializer. Several optimization methods were tried\footnote{EMF Performance Tips from EMF documentation can be found at \url{http://www.eclipse.org/modeling/emf/docs/performance/EMFPerformanceTips.html}}\footnote{Performance and Extensibility with EMF can be found at \url{http://www.slideshare.net/kenn.hussey/performance-and-extensibility-with-emf}} to no effect. We decided to test with smaller subsets of the test models (buildings) we had available.

Given the nature of the problems we experienced, and the large size of both the IFC meta model and building instance models, it would be appropriate to look into other modelling frameworks, or alternatively, attempt to minimise the size of the models used for testing. For example, one could test each floor of a building individually.

\paragraph{Valid Model Transformations} The Xtend-based model-to-model transformation from the partial IFC model to the Pipes model builds on a set of conversion rules. Validation of the transformation should therefore be reduced to the evaluation of these rules. We verify that each EClass of the target model has a corresponding rule. (TODO insert reference to tests) 

Similarly, the model-to-text transformation from the Pipes model instance to the .pipes format is based on a series of Xpand rules. The verification of this transformation is carried out as a part of the black box test described in \ref{subsec:verification_methods}, as the generated Pipes DSL editor will highlight any syntax errors in the Pipes DSL code.

\paragraph{A Simple DSL} Section \ref{subsec:pipes_dsl} describes the benefits of the particular DSL implementation of the prototype. It serves as a proof of concept that other subdomains of IFC can be edited in the same way. Being a modular component of the entire prototype, the textual Pipes DSL implementation could be substituted with a visual one.

\paragraph{Structural Editing} The structural editing on the partial model has, by far, been the most difficult unit of work in this project. Deleting elements is fairly simple, but a simple cleanup routine had to be implemented as the EMF utility still lets the model root element preserve a reference to the object even when marked for deletion by the user. Adding elements, on the other hand, is a complex task when working with an EMF model instance. (TODO write how we solved the problem of adding)


\subsection{Verification Methods}
\label{subsec:verification_methods}
Due to the nature of the prototype, an extensive and automated test suite for each workflow component is infeasible. Only the synchronisation features of the Main Model Updater component (see Figure \ref{fig:Pipes2IFCWorkflow}) is interesting to validate, as it is the central component containing most of the transformation logic.

Two approaches have been used to ensure the correctness of the prototype. Firstly, a test MWE2 workflow has been written to verify that the synchronisation logic in the Main Model Updater is done correctly. Secondly, a black box test of the entire prototype has been conducted with X (FIXME) input test buildings. These tests check that both the IFC model and the Pipes model is left in a valid state after running each workflow component. The results of the tests can be found in Appendix X (TODO insert ref).

\subsection{Threats to validity}
\paragraph{Internal Validity} When a prototype is primarily tested through black box testing, it is natural to question the correctness of all the individual components. However, we will argue that it is highly likely that any internal errors will result in either program crashes or inconsistencies discoverable by the black box tests, as the workflow components are highly modular with well defined interfaces.

Using EMF, with all its tools and languages, we have encountered an unusual amount of technical difficulties related only to the Eclipse IDE and its modelling tools. We identify these tools as an internal threat to the validity of the prototype, as our trust in them has been decreasing over the course of the project.

\paragraph{External Validity} Retrieving test buildings for the black box tests has been challenging. Given the fact that most modern visual BIM tools are very complex, we were unable to produce meaningful construction and plumbing models for black box testing, and had to rely on external contacts for providing test buildings (see Section \ref{sec:conclusion}). That means X (FIXME) example models were used for testing, which is considered sufficient but not optimal. More tests with more example models must be performed to solidify the claim of general applicability of the prototype to all building models.

Another concern might be that we only handle and test operations on our particular subdomain, but elements external to this domain are not paid any attention during the extraction and update processes, as they are expected to simply stay the same when untouched by our code. This assumption has not been officially confirmed.







