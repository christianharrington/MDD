%!TEX root = ./report.tex
\section{Evaluation}
\label{sec:evaluation}
In this section we evaluate the solution, with reference to each of the requirements in Section \ref{subsec:requirements}. Furthermore, we discuss the methods used to verify the correctness of the prototype.

% Maybe: Discussion of applicability of product. What should be done in order to make this project usable in a real world scenario?

\subsection{Requirements Evaluation}
\label{subsec:requirements_evaluation}
\subsubsection{Working with a Partial Model}
% By using Xtend classes to define workflow components, we are able to modularise the individual transformation steps. These steps are chained using MWE2 workflows. This gives reusability and verifiability of each of the modular components. Note in Figure \ref{fig:IFC2PipesWorkflow} and Figure \ref{fig:Pipes2IFCWorkflow} how several workflow components have been reused in both workflows.
We are capable of extracting partial IFC models, and subsequently update these with the Pipes DSL. In Section \ref{subsec:problem_analysis} we presented the informal IFC subset definition that we extract, and in in Figure \ref{fig:pipes_dsl_ecore_model} we saw the corresponding Pipes DSL meta model. We focus< on a simple partial model, but it would be possible to extend this if other IFC objects were to be added. By extracting the partial IFC model we have made subsequent transformations and work much easier and faster. As we have discussed, some issues are related with obtaining a good IFC meta model, so it makes good sense that the workflow component that directly depends on this can be seen in isolation.

\subsubsection{Correct Meta Model} As explained in Section \ref{subsec:ifc_meta_model} the prototype uses the official ifcXML XSD to generate an Ecore meta model that is both correct and easy to replace. However, there are certain serious drawbacks to this solution.

EMF does not seem to handle the large IFC meta models very well. In our experience, the IFC meta model generated from the IFC XSD, would be incomplete with certain methods not being auto-generated (eUnset methods for fields in some entity classes). Also, EMF will auto-generated methods that exceed the maximum size allowed by the Java Development Kit (64kB)\footnote{JDK bug 4262078: \url{http://bugs.sun.com/view_bug.do?bug_id=4262078}}. While it is possible to work around such problems, it is not viable the auto-generated code has to be fixed.

The default resource deserializer is too slow for working with an instance of a realistic IFC model. A real-world model can easily give load and save times at more than 10 minutes. The EMF site lists some performace tips that should meet these problems.\footnote{EMF Performance Tips: \url{http://www.eclipse.org/modeling/emf/docs/performance/EMFPerformanceTips.html}}\footnote{Performance and Extensibility with EMF: \url{http://www.slideshare.net/kenn.hussey/performance-and-extensibility-with-emf}}, but in our experience the improvement was minimal if any. In effect, all our tests have been are carried out using smaller subsets of IFC.

We conclude that EMF is not good for handling meta models of more than 2,000 elements.

\subsubsection{Valid Model Transformations} The Xtend-based model-to-model transformation from the partial IFC model to the Pipes model builds on a set of transformation rules. Validation of the transformation should therefore be reduced to evaluating these rules. We verify that each EClass of the target model has a corresponding rule (see Appendices \ref{app:blackboxtests} and \ref{app:automatedtests}).

Similarly, the model-to-text transformation from the Pipes model instance to the .pipes syntax is based on a series of Xpand rules. The verification of this transformation is carried out as a part of the black box test described in \ref{subsec:verification_methods}, as the generated Pipes DSL editor will highlight any syntax errors in the Pipes DSL code.

\subsubsection{A Simple DSL} Section \ref{subsec:pipes_dsl} describes the benefits of the particular DSL implementation of the prototype. It serves as a proof of concept that other subdomains of IFC can be edited in the same way. Being a modular component of the entire prototype, the textual Pipes DSL implementation could be substituted with a visual one.

\subsubsection{Structural Editing} The structural editing on the partial model has, by far, been the most difficult unit of work in this project. Deleting elements is fairly simple, but a simple cleanup routine had to be implemented as the EMF utility still lets the model root element preserve a reference to the object even when marked for deletion by the user. Adding elements to the model, however, turned out give rise to some serialization issues. Specifically, elements added to the model were serialized differently from other, similar ifcXML entities.

\subsection{Verification Methods}
\label{subsec:verification_methods}
Due to the nature of the prototype, an extensive and automated test suite for each workflow component is infeasible. Only IFC elements extraction, IFC2Pipes model transformation and the updating features of the Pipes2IFC Transformer components (see Figure \ref{fig:Pipes2IFCWorkflow}) are interesting to validate, as these are the central components containing most of the transformation logic. Furthermore, black box tests of the entire prototype have been conducted with X (FIXME) input test buildings. These tests check that both the IFC model and the Pipes model is left in a valid state after running each workflow component. The results of the tests can be found in Appendices \ref{app:blackboxtests} and \ref{app:automatedtests}.

\subsection{Threats to validity}
\subsubsection{Internal Validity} When a prototype is primarily black box tested, it is natural to question the correctness of all the individual components. However, we will argue that it is highly likely that any internal errors, undiscoverable by the black box tests, will either be discovered by the automated tests or result in program crashes, as the workflow components are highly modular with well defined interfaces.

Using EMF, with all its tools and languages, we have encountered an unusual amount of technical difficulties related only to the Eclipse IDE and its modelling tools. We identify these tools as an internal threat to the validity of the prototype, as our trust in them has been decreasing over the course of the project.

\subsubsection{External Validity} Retrieving test buildings for the black box tests has been challenging. Given the fact that most modern visual BIM tools are very complex, we were unable to produce meaningful construction and plumbing models for black box testing, and had to rely on external contacts for providing test buildings (see Section \ref{sec:conclusion}). That means X (FIXME) example models were used for testing, which is considered sufficient but not optimal. More tests with more example models must be performed to solidify the claim of general applicability of the prototype to all building models.

Another concern is that we only handle and test operations on our particular subdomain, but elements external to this domain are not paid any attention during the extraction, transformation and update processes, as they are expected to simply stay the same when untouched by our code. This assumption has not been officially confirmed.







