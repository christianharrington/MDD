%!TEX root = ./report.tex
% \pagenumbering{arabic}
% \setcounter{page}{1}
\section{Related Work}
%A Graphical User Interface for Handling IFC Partial Model exchange
Nour argues that the exchange of partial models is one of the central challenges of widespread use of a central model for BIM, mainly IFC\cite{nour08}. Although we do not adopt his idea of end-user filtering through the use of MVD, our approach does incorporate software filtering for defining a working subset of IFC. This subset has been defined in collaboration with Kaj J\o rgensen. For the central IFC data repository, BIMServer has been used\cite{beetz10}.

%Feature-Based Survey of Model Transformation Approaches
The model transformations used in our solution has mainly been realized through the direct manipulation approach described by Czarnecki et al.\cite{czarnecki06}, facilitated by the dispatch method feature of Xtend and the clarity of the Modeling Workflow Engine 2.

%Design guidelines for DSLs 
As the design of the concrete syntax for the presented DSL has been a minor aspect of our solution, it is far from perfect. The guidelines for DSL design presented by Karsai et al. \cite{karsai09} has been used as the basis for the preliminary design of the DSL, and further work in this direction will most likely benefit from the guidelines' focus on simplicity and domain concepts.

%From model transformation to incremental bidirectional model synchronization
%Combinators for bidirectional tree transformations: A linguistic approach to the view-update problem


%Embedded Software Development with Projectional Language Workbenches
%PrEdE: a Projectional Editor for the Eclipse Modeling Framework
An alternative to our approach using model synchronization with source editing is projectional editing. As specified by Tomasetti et al.\cite{tomasetti11}, projectional editors operate directly on the model, and thus no model synchronization is involved. An example of a projectional editor is JetBrains MPS. One advantage of this approach is that it is impossible to create inconsistent models. However, it also means a loss of notational flexibility, seeing that any notation must adhere directly to the model when no grammar or parser is involved\cite{conf/models/Voelter10}. Consequently, in order to be able to bring the notation as close to the domain as possible, a projectional editor has not been desirable for our solution.

%Towards a Framework for a Domain Specific Open Query Language for Building Information Models 




%BSPro COM-Server -- interoperability between software tools using industrial foundation classes

%Industry Foundation Classes And Interoperable Commercial Software In Support Of Design Of Energy-Efficient Buildings

