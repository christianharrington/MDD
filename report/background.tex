\section{Background}
\subsection{Building Information Modeling}
Building information modeling (BIM) is a method that through generating digital representations model physical constructions. The resulting model is a shared knowledge base that can be used by all the stages involving building management, such as the earliest conceptual stages, construction, maintenance, and even demolition. BIM extends traditional building design from two dimensional drawings and three dimensional physical models, to also include time and cost into the model, and even materials of specific parts. This makes BIM very attractive to modern building constructors, as it not only eases up the communication between the many different parts involved in a building, but also allows for early planning in terms of time- and energy consumption.
\subsection{Industry Foundation Classes}
Industry foundation classes (IFC) is a file format commonly used for BIM. Its goal is to facilitate interoperability between different software platforms. It does this through an object-based data model. More...

One of the main challenges when working with IFC is the sheer size of it. The entity model contains several hundred entities organized into an object-based inheritance hierarchy. Entities can be both tangible elements such as an IfcWall, but also abstract entities such as an IfcAxis2Placement3D describing the location and orientation of another IFC entity. On the highest level of abstraction, IFC defines two categories of elements, being root and unrooted elements. Unrooted elements do not have an identity and only exist if referred to by other elements. Rooted elements have a unique global identity (GUID) and are subdivided further into three groups: IfcObjectDefinitions are tangible elements, IfcRelations are relations between other objects, and IfcPropertyDefinitions are properties of other objects. The model is further subdivided multiple times. 

Ifc vs projectional editing.

Green Building XML (gbXML) is an alternative file format to IFC. A reason why IFC not gbXML?
\subsection{Workflow}
When modeling buildings, the structure of the building and its installations, such as plumbing and electrical, are modeled separately. The structure for example describe which walls are bearing and which walls have openings for electrical wiring or plumbing.  Because of this separate modeling, it is possible for installations to be inconsistent with the structure. I.e. a pipe or a electrical wire could be going through a solid wall without there being made an actual opening for it to go through. This could end up in a situation at the construction site where an installation isn't possible. It would therefore be desirable if the structural engineer could have an easy way to find these collisions and solve them, by i.e. making an opening or moving the installation.
\subsection{Eclipse Modeling Framework and Ecore}
The Eclipse modeling framework (EMF) is an Eclipse-based modeling framework that can be used to model a specific domain and generate tools for this domain. For describing a meta model for a domain, EMF includes Ecore. Ecore is a meta meta model, which is a model describing a meta model. Furthermore, it includes tools for transforming instances of these models to either text (Model to Text transformations or M2T) or other models (Model to Model transformations or M2M).