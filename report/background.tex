%!TEX root = ./report.tex
\section{Background}
\label{sec:background}
\subsection{Building Information Modeling}
\label{sec:building_information_modeling}
As previously described, BIM is a process that models physical constructions by generating digital representations. The resulting model is a shared knowledge base that can be used in all stages of a building project. BIM extends traditional building design from two-dimensional drawings and three-dimensional physical models, to include aspects such as time, cost, staffing, and design process. This makes BIM very attractive for modern building construction, as it not only eases the communication between the different stakeholders, but also allows for early planning in terms of time and energy consumption.

To support the flexibility required in BIM, this shared knowledge base has to be represented in a format that enables interoperability. And since building projects involve many different contributors from different domains, it has to be adoptable by many different software applications\,\cite{quteprints37725}. Consequently, an interoperability standard is needed.

\subsection{Industry Foundation Classes}
\label{sec:industry_foundation_classes}
IFC is widely used for BIM, and is mandatory for new public buildings in Denmark.\footnote{Regulations for buildings built by the Danish state: \url{https://www.retsinformation.dk/forms/R0710.aspx?id=134884}} Its goal is to facilitate interoperability between different software platforms\,\cite{steel11} and can reduce time spent on building design processes\,\cite{bazjanac99}. It does this through an object-based data model, represented in two file formats. The first is a EXPRESS-based format, IFC-EXPRESS (ISO-10303-21). The other is an XML format, called ifcXML (ISO-10303-28). The EXPRESS-based format is the most widely used due to its relative compact size, while still being readable.

One of the main challenges when working with IFC is that the models often are very large and complex. The meta model contains more than six hundred entities organised in an object-based inheritance hierarchy. Entities can be both tangible elements, such as an IfcWall, but also abstract entities like IfcAxis2Placement3D, describing the location and orientation of another IFC entity. On the highest level of abstraction, IFC defines two categories of elements: rooted and unrooted elements respectively. Unrooted elements do not have an identity and only exist if referred to by other elements. Rooted elements have a unique global identifier (globalId).\footnote{IFC2x3 Final Documentation: \url{http://www.buildingsmart-tech.org/ifc/IFC2x3/TC1/html/index.htm}}

\subsection{Eclipse Modeling Framework and Ecore}
The Eclipse Modeling Framework is a modelling framework and code generation facility for building tools and other applications, based on a structured data model.\footnote{The EMF project page can be found at \url{http://www.eclipse.org/modeling/emf/} and leads to MWE2, Xtext, Xtend, and Xpand resources.} EMF is capable of producing a Java object graph from a model instance described in XMI. For meta modelling, EMF features Ecore. Ecore is a meta modelling language allowing the user to describe and build meta models for their domain. Furthermore, several transformation tools, such as Xtend and Xpand, are associated with EMF. They support work with model transformations. Xtend is a programming language that compiles into Java. It is very well integrated into EMF, and has several language features that are well suited for model-to-model transformations. Xtend is implemented with Xtext, a tool allowing easy creation of DSLs. Xpand is a template language used in model-to-text transformations. EMF also features a workflow system, Modeling Workflow Engine 2 (MWE2), that makes it easy to chain these tools together.
