%!TEX root = ./report.tex
\section{Background}
\label{sec:background}
\subsection{Building Information Modeling}
\label{sec:building_information_modeling}
(FIXME Den første del her er allerede skrevet i starten af Introduction)BIM is a method that through generating digital representations, model physical constructions. The resulting model is a shared knowledge base that can be used in all the stages of a building project, ranging from the earliest conceptual stages, construction, to day to day management and even demolition. BIM extends traditional building design from two dimensional drawings and three dimensional physical models, to also include aspects like time, cost, manning and process into the model. This makes BIM very attractive to modern building constructors, as it not only eases up the communication between the many different parts involved in a building, but also allows for early planning in terms of time- and energy consumption.

To support the flexibility required in BIM, it has to be represented in a format that gives rise to interoperability. And since building projects involve many different contributers from different domains, it has to be adoptable by many different software applications\,\cite{quteprints37725}. In effect, most software applications in BIM projects, will only work on a subset of the model.
\subsection{Industry Foundation Classes}
IFC is commonly used for BIM and is mandatory for new public buildings in Denmark.\footnote{Regulations for buildings built by the Danish state \url{https://www.retsinformation.dk/forms/R0710.aspx?id=134884}}. Its goal is to facilitate interoperability between different software platforms. It does this through an object-based data model. There are two main file formats of IFC. The first is a EXPRESS based format, IFC-EXPRESS (ISO-10303-21), with the file extension ".ifc". The other is based on XML, and called ifcXML (ISO-10303-28), with the file extension ".ifcxml". The EXPRESS based format is the most commonly used due to its relative compact size, while still being readable.

One of the main challenges when working with IFC is the size of it. The entity model contains more than six hundred entities organized in an object-based inheritance hierarchy. Entities can be both tangible elements, such as an IfcWall, but also abstract entities like IfcAxis2Placement3D describing the location and orientation of another IFC entity. On the highest level of abstraction, IFC defines two categories of elements, being rooted and unrooted elements. Unrooted elements do not have an identity and only exist if referred to by other elements. Rooted elements have a unique global identity (GUID) and are subdivided further into three groups: IfcObjectDefinitions are abstract definitions of anything from tangible elements to definitions of work processes, IfcRelations are relations between other objects, and IfcPropertyDefinitions are properties of other objects. The model is further subdivided multiple times.

IFC is not the only file format used for BIM. Another prominent format is Green Building XML (gbXML) which also focuses on interoperability, and as the name suggests, lowering energy consumption. Although gbXML seems well evolved and flexible, for our purpose, IFC seems to be the best suited and also the most widely adopted.

\subsection{Eclipse Modeling Framework and Ecore}
The Eclipse Modeling Framework is a modeling framework and code generation facility for building tools and other applications based on a structured data model\footnote{The EMF project page can be found at \url{http://www.eclipse.org/modeling/emf/}}. EMF is capable of producing a Java object graph from a model instance described in XMI. For meta modeling this instance EMF includes Ecore. Ecore is a meta modeling language allowing the user to describe and build meta models for their domain. Furthermore EMF has several transformation tools, enabling transformation from either one Ecore model to another or from Ecore model directly to a textual format.
