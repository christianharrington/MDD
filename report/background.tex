%!TEX root = ./report.tex
\section{Background}
\label{sec:background}
In this section we provide an overview of BIM and some of the challenges in working with IFC. Lastly we discuss relevant technologies to be used for providing a solution.
\subsection{Building Information Modeling}
\label{sec:building_information_modeling}
As previously described, BIM is a method that through generating digital representations, model physical constructions. The resulting model is a shared knowledge base that can be used in all stages of a building project. BIM extends traditional building design from two-dimensional drawings and three-dimensional physical models, to also include aspects such as time, cost, manning, and process. This makes BIM very attractive for modern building constructors, as it not only eases up the communication between the different stakeholders, but also allows for early planning in terms of time and energy consumption.

To support the flexibility required in BIM, it has to be represented in a format that gives rise to interoperability. And since building projects involve many different contributors from different domains, it has to be adoptable by many different software applications\,\cite{quteprints37725}. In effect, most software applications in BIM projects will only work on a subset of the model.

\subsection{Industry Foundation Classes}
\label{sec:industry_foundation_classes}
IFC is commonly used for BIM and is mandatory for new public buildings in Denmark.\footnote{Regulations for buildings built by the Danish state \url{https://www.retsinformation.dk/forms/R0710.aspx?id=134884}}. Its goal is to facilitate interoperability between different software platforms. It does this through an object-based data model, represented in two main file formats. The first is a EXPRESS-based format, IFC-EXPRESS (ISO-10303-21). The other is based on XML, and called ifcXML (ISO-10303-28). The EXPRESS-based format is the most widely used due to its relative compact size, while still being readable.

One of the main challenges when working with IFC is that the models often get very big and complex. The meta model contains more than six hundred entities organised in an object-based inheritance hierarchy. Entities can be both tangible elements, such as an IfcWall, but also abstract entities like IfcAxis2Placement3D describing the location and orientation of another IFC entity. On the highest level of abstraction, IFC defines two categories of elements, those being rooted and unrooted elements respectfully. Unrooted elements do not have an identity and only exist if referred to by other elements. Rooted elements have a unique global identifier (globalId) and are subdivided further into three groups: IfcObjectDefinitions are abstract definitions of anything from tangible elements to definitions of work processes, IfcRelations are relations between objects, and IfcPropertyDefinitions being properties of objects. We refer to the literature for an overview of IFC.

\subsection{Eclipse Modeling Framework and Ecore}
The Eclipse Modeling Framework is a modelling framework and code generation facility for building tools and other applications based on a structured data model\footnote{The EMF project page can be found at \url{http://www.eclipse.org/modeling/emf/}}. EMF is capable of producing a Java object graph from a model instance described in XMI. For meta modelling EMF features Ecore. Ecore is a meta modeling language allowing the user to describe and build meta models for their domain. Furthermore, EMF has several transformation tools such as Xtend and Xpand, that supports work with model transformations.
