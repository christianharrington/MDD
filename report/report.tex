%!TEX TS-options = -shell-escape
\documentclass[oribibl]{llncs}
\usepackage{makeidx}  % allows for indexgeneration

\usepackage[utf8]{inputenc}
\usepackage{pdfpages}

% \usepackage[nottoc,numbib]{tocbibind}
% \usepackage[center]{caption}

\usepackage{enumitem}


% %% to make urls look better in bibliography
% \makeatletter
% \def\url@leostyle{%
%   \@ifundefined{selectfont}{\def\UrlFont{\sf}}{\def\UrlFont{\small\ttfamily}}}
% \makeatother
% %% Now actually use the newly defined style.
% \urlstyle{leo}

\bibliographystyle{plain}



\begin{document}

% insert the table of contents if want, it is not required
% llncs says: "If you are the author of a single contribution you
% normally have no running heads and no table of contents."
% \tableofcontents

\mainmatter              % start of the contributionsmainmatter
\title{DSL for Textual Editing of IFC partial model\thanks{Supervised by Andrzej Wasowski, IT University of Copenhagen, Denmark}}

\author{Nicolai Dahl Blicher-Petersen \and Christian Harrington \and
Thomas Hallier Didriksen \and Sune Alkærsig \and Anders Høst Kjærgaard\inst{1}\\\email{\{ndbl, cnha, thdi, sual, ahkj\}@itu.dk}}

% Do we want to show our emails?
\institute{IT University of Copenhagen, Rued Langgaards Vej 7, 2300 Copenhagen S, Denmark}


\maketitle              % typeset the title of the contribution

\begin{abstract}
Industry Foundation Classes (IFC) are often big and complex, and a key challenge is how work can be done on a partial model or aspect - such as heating or electrical wiring. This is a challenge for IFC projects in general, as projects often involve various stakeholders, all interested in only a part of the IFC relevant to their sub-domain. In this paper we demonstrate how the model-driven methodology can support the extraction of a partial model from the IFC model, transform it to a DSL (Domain Specific Language) for editing, and synchronize it back to the IFC model. We argue that our approach can leverage the adoption of IFC in emerging domains.


\keywords{IFC, DSL, model-driven development, model transformation}
\end{abstract}

% Number the first pages with Roman numbers
\pagenumbering{roman}

%!TEX root = ./report.tex


% Slides:
% - General statement introducing the area; You can most likely start with the first paragraph from your project description and evolve it.
% - Explanation of the specific problem and why do we care about the problem.
% - Explanation of your solution, and how it improves on the work by others. Relation to related work can be very brief, given that you have a separate extensive section devoted to this.
%  -A hint on how the solution was evaluated and what was the outcome of this evaluation.
%  -A summary (a “map”) of how the paper is organized.

\pagenumbering{arabic}
\setcounter{page}{1}
\section{Introduction}
Building Information Modelling (BIM) is the process of modelling various physical and functional aspects of a building\cite{clar:eke}. Currently, many BIM software products exist. These let a user model and analyze the many interactions between different parts of a building. Many of these software products use common file formats for modeling buildings, such as Industry Foundation Classes (IFC) or Green Building XML (gbXML). These formats were invented as industry standards to help improve interoperability between different software packages in different domains, and as such cover many different systems.
\paragraph{}
Reading and writing huge, complex XML documents, like the ones produced by BIM products, by hand is unfeasible. It’s desirable to be able to work with aspects of a model (such as heating, electrical wiring or plumbing) independently.(TODO insert reference) In this paper we present a setup, demonstrating that such projectional editing is feasible using modern modelling tools.
\paragraph{}
Specifically, this paper focusses on solving a common problem of combining a construction model with a plumbing model in a managable way. The solution entails a central server storing a merged version of these models and a client side enabling a user to edit and verify the correctness of the merging in a domain specific language (DSL), called Pipes DSL, in a Xtext-generated Eclipse editor, making sure that the construction is prepared to handle the plumbing installations. The solution is extensible in several ways making it a good starting point for future automatization and visualization.
\paragraph{}
This paper first gives a background of the problem space before laying out a list of desirable features for the end product. The solution is described and then evaluated with test buildings and reference to the desirable features. Finally, we present a list of future work to be carried out as well as a conclusion.



%!TEX root = ./report.tex
\section{Background}
\label{sec:background}
\subsection{Building Information Modeling}
\label{sec:building_information_modeling}
As previously described, BIM is a method that through generating digital representations, model physical constructions. The resulting model is a shared knowledge base that can be used in all stages of a building project. BIM extends traditional building design from two-dimensional drawings and three-dimensional physical models, to also include aspects such as time, cost, manning, and process. This makes BIM very attractive for modern building constructors, as it not only eases up the communication between the different stakeholders, but also allows for early planning in terms of time and energy consumption.

To support the flexibility required in BIM, it has to be represented in a format that gives rise to interoperability. And since building projects involve many different contributors from different domains, it has to be adoptable by many different software applications\,\cite{quteprints37725}. In effect, most software applications in BIM projects will only work on a subset of the model.

\subsection{Industry Foundation Classes}
\label{sec:industry_foundation_classes}
IFC is commonly used for BIM and is mandatory for new public buildings in Denmark.\footnote{Regulations for buildings built by the Danish state \url{https://www.retsinformation.dk/forms/R0710.aspx?id=134884}}. Its goal is to facilitate interoperability between different software platforms. It does this through an object-based data model, represented in two main file formats. The first is a EXPRESS-based format, IFC-EXPRESS (ISO-10303-21). The other is based on XML, and called ifcXML (ISO-10303-28). The EXPRESS-based format is the most widely used due to its relative compact size, while still being readable.

One of the main challenges when working with IFC is that the models often get very big and complex. The meta model contains more than six hundred entities organised in an object-based inheritance hierarchy. Entities can be both tangible elements, such as an IfcWall, but also abstract entities like IfcAxis2Placement3D describing the location and orientation of another IFC entity. On the highest level of abstraction, IFC defines two categories of elements, those being rooted and unrooted elements respectfully. Unrooted elements do not have an identity and only exist if referred to by other elements. Rooted elements have a unique global identifier (globalId) and are subdivided further into three groups: IfcObjectDefinitions are abstract definitions of anything from tangible elements to definitions of work processes, IfcRelations are relations between objects, and IfcPropertyDefinitions being properties of objects. We refer to the literature for an overview of IFC.

\subsection{Eclipse Modeling Framework and Ecore}
The Eclipse Modeling Framework is a modelling framework and code generation facility for building tools and other applications based on a structured data model\footnote{The EMF project page can be found at \url{http://www.eclipse.org/modeling/emf/}}. EMF is capable of producing a Java object graph from a model instance described in XMI. For meta modelling EMF features Ecore. Ecore is a meta modeling language allowing the user to describe and build meta models for their domain. Furthermore, EMF has several transformation tools such as Xtend and Xpand, that supports work with model transformations. Xtend is a programming language that compiles into Java. It is very well integrated into EMF, and has several language features that are well suited for model-to-model transformations, such as dispatch and create methods. Xtend is implemented with Xtext, a tool allowing for easy creation of DSLs. Xpand is template language used in model to text transformations. EMF also features a workflow system, MWE2, that makes it easy to chain these tools together.

%!TEX root = ./report.tex
\section{Desirable Features}
In the following we will discuss a set of desirable features for a solution, that can edit a partial IFC model.
As stated earlier, a key challenge in working with IFC, is the ability to work on a partial model, and eventhough a variety many advanced implementations exist, most of them have mainly been focused on graphical editing for architectural domains, leaving out a world of domains that are being involved with the IFC. Also, across the existing implementation, it is not at all consistent how models are treated.[Model interoperability in Building Information Modelling]. The features we list here focus on how we in general can go from the complex IFC to a simple editing environment.

\subsection{Working with a Partial model}
With the before mentioned complexities and challenges of IFC in mind, a primary focus is to be able to extract a well specified
subset of the IFC.  We want an architecture that separates this concern from the rest of the workflow into one component, that extracts the
partial model that we are interested in. Reversely the concern of re-inserting this partial model we can also
see as a workflow component in isolation. It is not at all trivial to define how such a partial model is extracted, thus, it is a concern we
don't want to take account for at other stages of the workflow. An evolved standard for this kind of operation is by
Model View Definition(MVD)\cite(MVD text).\cite(Muhammed fra Tyskland project)  describes, what seems to be, a promising MVD
implementation for IFC, that allows for fine-grained XML definitions of a partial IFC model. It is yet to be obtained and tested on our part.
(How else do we say we know about this, but that we did'nt use it? Also notes say something about using BimServer for getting the partial model, how?) However, defining a partial model for the MVD seems, to our experience, in itself not as a trivial or easy task, so making this step simpler would be ideal. Also, as the IFC standard is continuously updated, it would be desirable to be able to replace the IFC meta model when it is updated.
    In the solution we want a well defined subset of the model, that, as a minimum, defines what is of interest in the target model, and what is it needed
to make it possible to synchronize back to the source at a later stage.

\subsection{Valid Model to Model Transformation (M2M)}
An ideal solution would produce model transformations that are verifiable and correct. By verifiable, we refer to being able to trace that transformations actually occur in the way that we require. By correct, we primarily refer to not corrupting any model structure during a transformation, but also to not breaking any constraints in the domains. However, with the complexity of IFC in mind, we will relax the requirement on not breaking any constraints. As such, the ideal solution would show that a making a valid M2M is feasible, but we accept that some IFC constraints might have to be accounted for in a real-world implementation.

\subsection{A Simple DSL}
Key to the solution, is that is shows a part of the complex source domain can be represented and edited in a simple way. Targeting a simple DSL model would show this, and it will demonstrate that the target model can easily be exchanged with a new one. The latter is important, as we can imagine that future target domains, may be any arbitrary sub-domain of IFC or graphical editors. Again, we are not aiming at a rich and flexible DSL, but rather a simple language that we can easily modify. For our solution, we would like to show a target domain where we work on simple notions like the position of a pipe and a hole in a wall. To support this in being extensible and reusable, it should be build with wide-spread tools like EMF and Xtext.

\subsection{Structural Editing}
The DSL should allow for structural editing.

\subsection{Correct Meta Model}
It is vital for the solution that we work on a correct meta model and model instances of the IFC. The meta model can be obtained from several sources,
and is of course required that we are able to port an instance of this to ecore. Accordingly we want to be able to serialize and deserialize IFC model instances from IFC files, that, for the most part, exist in the EXPRESS format or IFCXML. The ideal solution would reuse an existing solution for this, where http://www.ifcwiki.org/index.php/Open_Source lists a good handful.


In this section we present a list of desirable features when working with projectional editing on IFC.
\paragraph{}
Design decisions we faced when making this setup. List a bunch of requirements that we can reference in the next section:
\begin{enumerate}[itemsep=0pt,parsep=2pt,topsep=10pt]
	\item {\it Full control over projection software.} Write about: Extracting subset on server og client? Why not trying to do this on the BIMServer? We could have extended the open source project
	\item {\it Accurate ecore meta-model for IFC domain.} Using BIMServer library ecore metamodel or using XSD generated ecore metamodel
	\item {\it Proper subset definition.} How is the subset defined? MVD could have been used, but we chose not to.
	\item {\it Flexible model to model framework.} Which model2model tool to use? ATL, QVT, Xtend?
\end{enumerate}

%!TEX root = ./report.tex
\section{Solution}
The solution combines several technologies to attain an extensible solution with two reusable Xtend client side workflows at its core. The setup allows the user to edit a subset of IFC in a generated Eclipse editor in between the two workflows. The main model is stored on a BIMServer for merging, versioning and extensibility purposes, as this section will further explain.

\subsection{Server Side}
The overall workflow of the end product is depicted in Figure \ref{fig:overall_product_workflow}. As described in Kaj Jørgensen's workflow document\cite{jorgensen12} a construction model and a plumbing model are combined into one single model that needs to be verified for consistency. So called openings, i.e. holes in walls and floors need to be in place where the plumbing model describes flow segments to be installed. The merging of these models is executed on the BIMServer by the user as the first step of the workflow.\cite{bimserver} When merging has finished, the client side of the solution can retrive the merged building as XML. The user is now allowed to edit and add elements to the subset of the model on the client side before saving the building back to the BIMServer as XML.

Note that the solution does not take concurrent editing by multiple clients into account although the BIMServer does support version control as well as merging. We will leave it for future work to ensure correct handling of current editing by clients.

\begin{figure}[htbp]
    \centering
        \includegraphics[width=120mm]{images/CompleteWorkflow.pdf}
    \caption{Overall Product Workflow}
    \label{fig:overall_product_workflow}
\end{figure}

We let a BIMServer handle the initial merging of plumbing and construction models although many tools exist for this job. The special advantage of the BIMServer is that it also provides, conversion tools allowing the user to retrieve the saved building in other formats as well as Java client library in addition to the above mentioned version control features. In particular the Java client library is important for automating the workflow, as it allows for programatic server communication. (TODO maybe add a brief discussion of the alternatives?)

\subsection{Client Side}
Figure \ref{fig:IFC2PipesWorkflow} shows the IFC to Pipes workflow retrieving an IFC model from the BIMServer as XML, processing it to the corresponding Java object graph, extracting the pipes and opening elements and converting these to an editable DSL instance, which is in turn saved to disk as an XMI file. The XML file loaded from the server is saved to the local disk for use by the second workflow explained below.

\begin{figure}[htbp]
    \centering
        \includegraphics[width=120mm]{images/IFC2Pipes.pdf}
    \caption{IFC to Pipes DSL workflow}
    \label{fig:IFC2PipesWorkflow}
\end{figure}

The extraction process where relevant IFC elements are collected from the main model for conversion is implemented by a simple filtering mechanism extracting the pipe and opening elements. The real work of the IFC to Pipes workflow is done when this extract is converted to the corresponding Pipes DSL instance in the second to last step of the workflow. By utilizing the convenient model to model transformation language features of the Xtend language, each object is transformed from the IFC model to the corresponding PipesDSL object. Every IFC object inside the solution domain specified in Section \ref{label:solution_domain_definition} has a transformation rule specified here, making the conversion possible.

The second client side Xtend workflow, Pipes to IFC, is depicted on Figure \ref{fig:Pipes2IFCWorkflow}, where the user-edited XMI file is loaded into a Main Model Updater workflow module together with the non-updated extracted instance of the main model. Notice how the extraction is loaded in using the same workflow modules as in the first workflow, except the XML file is not fetched from the server but from the local disk. This makes for an easier update process in the Main Model Updater as the extracted instance is guaranteed not to have changed while the user was editing the XMI file. After the main model instance has been updated to reflect the users changes it is converted to XML and this new file is saved back to the BIMServer.

\begin{figure}[htbp]
    \centering
        \includegraphics[width=120mm]{images/Pipes2IFC.pdf}
    \caption{Pipes DSL to IFC workflow}
    \label{fig:Pipes2IFCWorkflow}
\end{figure}

Clearly, most of the work in this workflow lies with the Main Model Updater, which uses a pattern match-like dispacth language feature of Xtend to elegantly update the java object graph of the extracted model. The routine runs through all elements in the Java object graph to update the corresponding Pipes DSL objects by looking at matching GUIDs. Only traversing an extract of the main model guarantees a better running time and still guarantees a correct update procedure, as the Pipes DSL does not allow any updates to elements outside of this extract anyway.
\subsubsection{Adding and Removing}
The setup not only supports updating the attributes of elements, it also allows the user to do structural changes on the model. The Main Model Updater looks for elements in the Pipes DSL object graph without a GUID to determine if a new element should be created in the IFC model. This feature allows the user to add missing openings for pipe segments and is therefore crucial for the usefulness of the end product. 

Likewise, if the updater fails to find an element in Pipes DSL object graph that corresponds to the existing one in the Java object graph it means the element has been deleted and that the Main Model Updater should take appropriate action to update references.

\subsection{Pipes DSL}
An example or two about the DSL and why it simplifies the work with the Pipes/Openings domain.













 %Describe the solution
%!TEX root = ./report.tex
\section{Evaluation}
\label{sec:evaluation}
In this section we evaluate the solution, with reference to each of the requirements in Section \ref{subsec:requirements}. Furthermore, we discuss the methods used to verify the correctness of the prototype.

% Maybe: Discussion of applicability of product. What should be done in order to make this project usable in a real world scenario?

\subsection{Requirements Evaluation}
\label{subsec:requirements_evaluation}
\subsubsection{Working with a Partial Model}
We are capable of extracting partial IFC models, and subsequently updating these by using the Pipes DSL. In Section \ref{subsec:problem_analysis} we presented the informal IFC subset definition that we extract, and in Figure \ref{fig:pipes_dsl_ecore_model} we depicted the corresponding Pipes DSL meta model. We focus on a simple partial model, but it would be possible to extend this if other IFC objects were to be added. We have ensured modularity by implementing the extraction process as a modular workflow component. Hence, the extraction process is verifiable as confirmed by the automated tests described in Section \ref{subsec:verification_methods}.

\subsubsection{Correct Meta Model} As explained in Section \ref{subsec:ifc_meta_model} the prototype uses the official ifcXML XSD to generate an Ecore meta model that is both correct and replaceable in theory. However, there are certain serious drawbacks to this solution.

EMF does not seem to handle the large IFC meta models very well. In our experience, the IFC meta model generated from the IFC XSD, would be incomplete with certain methods not being auto-generated (eUnset methods for fields in some entity classes). Also, EMF will auto-generated methods that exceed the maximum size allowed by the Java Development Kit (64kB)\footnote{JDK bug 4262078: \url{http://bugs.sun.com/view_bug.do?bug_id=4262078}}. While it is possible to work around such problems, it is not viable the auto-generated code has to be fixed.

The default resource deserialiser is too slow for working with an instance of a realistic IFC model. A real-world model can easily give load and save times of more than ten minutes. The EMF site lists some performance tips that should solve these problems.\footnote{EMF Performance Tips: \url{http://www.eclipse.org/modeling/emf/docs/performance/EMFPerformanceTips.html}}\footnote{Performance and Extensibility with EMF: \url{http://www.slideshare.net/kenn.hussey/performance-and-extensibility-with-emf}}, but in our experience the improvement was minimal if any. In effect, all our tests have been are carried out using smaller subsets of IFC. We conclude that EMF is not good for handling meta models of more than 2,000 elements.

\subsubsection{Valid Model Transformations} The Xtend-based model-to-model transformation from the partial IFC model to the Pipes model builds on a set of transformation rules. Validation of the transformation should therefore be reduced to evaluating these rules. We verify that the transformation on rules for each of the objects in the target model is correct (see Appendix \ref{app:automatedtests}).

Similarly, the model-to-text transformation from the Pipes model instance to the .pipes syntax is based on a series of Xpand rules. The verification of this transformation is carried out as a part of the black box test described in \ref{subsec:verification_methods} and Appendix \ref{app:blackboxtests}, as the generated Pipes DSL editor will highlight any syntax errors in the Pipes DSL code.

\subsubsection{A Simple DSL} Section \ref{subsec:pipes_dsl} describes the benefits of the DSL implementation. It serves as a proof of concept that other subdomains of IFC can be edited in the same way. Being a modular component of the entire prototype, the textual Pipes DSL implementation could be substituted with a visual one.

\subsubsection{Structural Editing} The structural editing on the partial model has, by far, been the most difficult unit of work in this project. Deleting elements is fairly simple, but a simple cleanup routine had to be implemented as the EMF still lets the model root element preserve a reference to the object even when marked for deletion. Adding elements to the model, however, turned out give rise to some serialisation issues. Specifically, elements added to the model were serialised differently from other, similar ifcXML entities.

\subsection{Verification Methods}
\label{subsec:verification_methods}
We have implemented an automated test suite for verifying that our transformation components work as desired. The IFC elements extractor, IFC2Pipes model transformation and the updating features of the Pipes2IFC Transformer components (see Figure \ref{fig:Pipes2IFCWorkflow}) are the most interesting to validate, as these are the central components containing most of the transformation logic. Furthermore, black box tests of the entire prototype have been conducted with X (FIXME) input test buildings. These tests check that both the IFC model and the Pipes model is left in a valid state after running each workflow component. The results of the tests can be found in Appendices \ref{app:blackboxtests} and \ref{app:automatedtests}.

\subsection{Threats to validity}
\subsubsection{Internal Validity} When a prototype is primarily black box tested, it is natural to question the correctness of all the individual components. However, we will argue that it is highly likely that any internal errors, undiscoverable by the black box tests, will either be discovered by the automated tests or result in program crashes, as the workflow components are highly modular with well defined interfaces.

Using EMF we have encountered a substantial amount of technical difficulties related only to the Eclipse IDE and its modelling tools. We identify these tools as an internal threat to the validity of the prototype, as our trust in them has been decreasing over the course of the project.

\subsubsection{External Validity} Retrieving building models for testing has been challenging. Given the fact that most modern visual BIM tools are very complex, we were unable to produce meaningful construction and plumbing models for testing ourselves, and had to rely on external contacts for providing test models (see Section \ref{sec:conclusion}). That means X (FIXME) example models were used for testing, which is considered sufficient, but not optimal. More tests with more example models must be performed to solidify the claim of general applicability of the prototype to all building models.








%!TEX root = ./report.tex
\section{Future Work}
\label{sec:future_work}
\subsection{Meta model}
\subsection{Synchronization}
% Study these papers:
%	"Feature-Based Survey of Model Transformation Approaches",
%	"The View Update Problem for XML", 
%	"From model transformation to incremental bidirectional model synchronization" (very technical paper)
\subsection{DSL}
%!TEX root = ./report.tex
\bibliography{report}
%!TEX root = ./report.tex
\section{Conclusion}
\paragraph{Acknowledgements}
TODO write this section: People that did proof reading. Andrzej our supervisor. Kaj Jørgensen for providing example buildings as well guidance in selecting a fitting IFC domain for our project. Mathias Demant for example buildings for test data.
%!TEX root = ./report.tex
\bibliography{report}

%% Old bibliography
% %
% % ---- Bibliography ----
% %
% \begin{thebibliography}{99}
% %
% \bibitem[1]{clar:eke}
% Clarke, F., Ekeland, I.:
% Nonlinear oscillations and
% boundary-value problems for Hamiltonian systems.
% Arch. Rat. Mech. Anal. 78, 315--333 (1982)

% \bibitem[2]{2clar:eke:2}
% Clarke, F., Ekeland, I.:
% Solutions p\'{e}riodiques, du
% p\'{e}riode donn\'{e}e, des \'{e}quations hamiltoniennes.
% Note CRAS Paris 287, 1013--1015 (1978)

% \bibitem[3]{2mich:tar}
% Michalek, R., Tarantello, G.:
% Subharmonic solutions with prescribed minimal
% period for nonautonomous Hamiltonian systems.
% J. Diff. Eq. 72, 28--55 (1988)

% \bibitem[4]{2tar}
% Tarantello, G.:
% Subharmonic solutions for Hamiltonian
% systems via a $\bbbz_{p}$ pseudoindex theory.
% Annali di Matematica Pura (to appear)

% \bibitem[5]{2rab}
% Rabinowitz, P.:
% On subharmonic solutions of a Hamiltonian system.
% Comm. Pure Appl. Math. 33, 609--633 (1980)

% \bibitem[6]{nour08}
% Nour, M.:
% A Graphical User Interface for handling IFC Partial Model Exchange

% \end{thebibliography}


\end{document}









