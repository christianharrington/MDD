%!TEX root = ./report.tex

% Slides:
% - General statement introducing the area; You can most likely start with the first paragraph from your project description and evolve it.
% - Explanation of the specific problem and why do we care about the problem.
% - Explanation of your solution, and how it improves on the work by others. Relation to related work can be very brief, given that you have a separate extensive section devoted to this.
%  -A hint on how the solution was evaluated and what was the outcome of this evaluation.
%  -A summary (a “map”) of how the paper is organized.

\pagenumbering{arabic}
\setcounter{page}{1}
\section{Introduction}
Building Information Modelling (BIM) is the process of modelling various physical and functional aspects of a building. Currently, many BIM software products exist. These allow users to model and analyse the many interactions between different parts of a building. Many of these software products use common file formats for modeling buildings, such as Industry Foundation Classes (IFC) or Green Building XML (gbXML). These formats were invented as industry standards to help improve interoperability between different software packages in different domains, and as such cover many different systems.

In a real-life scenario engineers usually work on separate parts of a building model using advanced visual tools such as Revit\footnote{An overview of Revit products can be found at \url{http://usa.autodesk.com/revit/}}. Due to the complexity of most BIM projects, the IFC files grow big and incomprehensible for humans to read or edit, and effectively a challenge for software that wants to work with the models. As described in Section \ref{sec:building_information_modeling} it is desirable to be able to work with aspects of a model such as heating, electrical wiring or plumbing independently. In this paper we present a setup, that demonstrates that such editing on a subset of an IFC model is feasible using modern modeling tools like Eclipse Modeling Foundation (EMF), Xtend and Xtext.

\paragraph{Problem}
In particular, we focus on solving a common problem\,\cite[pp. 20]{jorgensen10} of combining a construction model with a plumbing model in a manageable way. When modeling buildings, the structure of the building and its installations, such as plumbing and electrical, are modeled separately\,\cite[pp. 19--20]{jorgensen10}. The structure describes which walls are bearing and which walls have openings for electrical wiring or plumbing.  Because of this separate modeling, it is possible for installations to be inconsistent with the structure. For example, a pipe or an electrical wire could intersect a solid wall, without an opening for it to go through. This could result in a situation at the construction site where a planned installation is not possible in reality. 

\paragraph{Solution}
Due to the difficulties mentioned above, it would be desirable if the construction engineer had an easy way to find these collisions and solve them, by making an opening or moving the installation\,\cite{jorgensen12}. To facilitate this, we will…
\begin{itemize}
\item Produce an analysis of the subset of IFC used to define pipe installations, and develop a simple, textual domain specific language (DSL), called Pipes DSL, to specify these installations.
\item Develop an Eclipse Editor that eases the writing of files with this DSL, including syntax highlighting and autocompletion.
\item Create a synchronization mechanism between the DSL and the IFC model being manipulated, so changes made in the DSL are applied correctly to the model.
\item Interface with a central server, that enables the storing and merging of the models. This frees the editor from some responsibilities, and enables the possibility of collaborative work.
\item Plan for future BIM-related projects in the Model Driven Development course, based on this project.
\end{itemize}

\paragraph{Workflow}
\label{sec:workflow}
A model is retrieved on the client side from the server, and the relevant parts are extracted and transformed to the Pipes DSL. The editor enables the user to edit and verify the data and structure in the model. These changes are synchronized back to the main model, and stored server-side. The solution is extensible and constitutes as a good starting point for future automation and visualization, which will be discussed in Section \ref{sec:future_work}.

\paragraph{}
This paper first gives a background of the problem space before presenting a list of desirable features for a solution trying to accomplish partial editing on a large model such as IFC. The solution is described and then evaluated with test buildings and reference to the desirable features. Finally, we suggest future work and conclude on our findings.
