 %!TEX root = ./report.tex


% Slides:
% - General statement introducing the area; You can most likely start with the first paragraph from your project description and evolve it.
% - Explanation of the specific problem and why do we care about the problem.
% - Explanation of your solution, and how it improves on the work by others. Relation to related work can be very brief, given that you have a separate extensive section devoted to this.
%  -A hint on how the solution was evaluated and what was the outcome of this evaluation.
%  -A summary (a “map”) of how the paper is organized.

\pagenumbering{arabic}
\setcounter{page}{1}
\section{Introduction}
Building Information Modelling (BIM) is the process of modelling various physical and functional aspects of a building. Currently, many BIM software products exist. These allow users to model and analyse the many interactions between different parts of a building. Many of these software products use common file formats for modeling buildings, such as Industry Foundation Classes (IFC) or Green Building XML (gbXML). These formats were invented as industry standards to help improve interoperability between different software packages in different domains, and as such cover many different systems.

In a real-life scenario engineers usually work on separate parts of a building model using advanced visual tools such as Revit\cite{revit12}. Due to the complexity of most BIM projects, the IFC files grow big and incomprehensible for humans to read or edit, and effectively a challenge for software that wants to work with the models. As described in Section \ref{sec:building_information_modeling} it is desirable to be able to work with aspects of a model such as heating, electrical wiring or plumbing independently. In this paper we present a setup, that demonstrates that such editing on a subset of an IFC model is feasible using modern modeling tools like Eclipse Modeling Foundation (EMF), Xtend and Xtext.

In particular, we focus on solving a common problem (\cite{jorgensen10} pp. 20) of combining a construction model with a plumbing model in a manageable way. The solution includes a central server for merging and storing a version of the main models, and a client side enabling the user to edit and verify the data and structure in the target model, that will be synchronized back to the main model, and stored server-side. This is accomplished by providing an Eclipse editor allowing the user to get an overview of a subset of the building model in a domain specific language (DSL), called Pipes DSL. The user is able to verify that the construction is prepared to handle the plumbing installations, and apply adjustments if needed. The solution is extensible and constitutes as a good starting point for future automation and visualization, which will be discussed in Section \ref{sec:future_work}.

This paper first gives a background of the problem space before presenting a list of desirable features for a solution trying to accomplish partial editing on a large model such as IFC. The solution is described and then evaluated with test buildings and reference to the desirable features. Finally, we suggest future work and conclude on our findings.

\subsection{Acknowledgements}
TODO write this section: People that did proof reading. Andrzej our supervisor. Kaj Jørgensen for providing example buildings as well guidance in selecting a fitting IFC domain for our project. Mathias Demant for example buildings for test data.

