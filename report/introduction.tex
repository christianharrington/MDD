%!TEX root = ./report.tex

% Slides:
% - General statement introducing the area; You can most likely start with the first paragraph from your project description and evolve it.
% - Explanation of the specific problem and why do we care about the problem.
% - Explanation of your solution, and how it improves on the work by others. Relation to related work can be very brief, given that you have a separate extensive section devoted to this.
%  -A hint on how the solution was evaluated and what was the outcome of this evaluation.
%  -A summary (a “map”) of how the paper is organized.

\pagenumbering{arabic}
\setcounter{page}{1}
\section{Introduction}
Building Information Modelling (BIM) is the process of modelling various physical and functional aspects of a building. Currently, many BIM software products exist. These allow users to model and analyse the many interactions between different parts of a building. Many of these software products use common file formats for modelling buildings, such as Industry Foundation Classes (IFC). This format improves interoperability between different software packages in different domains, and is integrated into a variety of commercial applications.

In a real-life scenario, engineers usually work on separate parts of a building model using advanced, proprietary, and visual tools such as Revit\footnote{An overview of Revit products can be found at \url{http://usa.autodesk.com/revit/}}. It is desirable to be able to work with aspects of a model such as heating, electrical wiring, or plumbing independently due to the complexity of most BIM projects. In this paper we present a prototype that demonstrates that such editing on a subset of an IFC model is feasible using modern modelling tools from the Eclipse Modelling Foundation (EMF).

\subsubsection{Problem}
In particular, we focus on solving a common problem of combining a construction model with a plumbing model in a manageable way. When modelling buildings, the structure of the building and its plumbing installations are modelled separately\,\cite[pp. 19--20]{jorgensen10}. The construction model describes which walls are bearing and which walls have openings for installations. Because of this separate modelling, it is possible for plumbing installations to be inconsistent with the construction model. For example, a pipe could intersect with a solid wall, without an opening for it to go through. This could result in a situation at the construction site where a planned installation is not possible in reality. This problem has been defined by Kaj A. Jørgensen, associate professor at Aalborg University, who has substantial domain knowledge through his work with IT methodologies for building modelling\footnote{Kaj Jørgensens page at Aalborg University: \url{http://www.kaj.person.aau.dk/}}.

\subsubsection{Solution}
Due to the difficulties mentioned above, it would be desirable if the construction engineer had an easy way to find these collisions and solve them, by making an opening or moving the installation\,\cite{jorgensen12}. To facilitate this, we will…
\begin{itemize}
\item Produce an analysis of the subset of IFC used to define pipe installations, and develop a simple, textual domain specific language (DSL), called Pipes, to specify these installations.
\item Develop an Eclipse editor that eases the writing of files with the DSL, including syntax highlighting and autocompletion.
\item Create an update mechanism between the DSL and the IFC model being manipulated, so changes made in the DSL are applied correctly to the model.
\item Interface with a central server for storing, merging and (future) collaboration support. We use BIMServer\footnote{BIMServer Documentation can be found at \url{http://bimserver.org/documentation/}}, which is open source.
\item Present ideas for future BIM projects in relation to the ITU research initiative, Energy Futures, based on this project.\footnote{For further information, see \url{http://energyfutures.itu.dk/}}
\end{itemize}

\subsubsection{Workflow}
\label{sec:workflow}
The workflow revolves around a server. We save a construction and plumbing model at the BIMServer, which then merges the two. This merged IFC model is retrieved by our client, and the relevant parts are extracted and transformed to the Pipes DSL syntax. An editor for the Pipes DSL is launched, and the user can edit and verify the consistency of its data and structure. Finally, these changes are updated in the merged model and stored server-side. The solution is extensible and constitutes a good starting point for future automation of the workflow and visualization of the domain.

\paragraph{}
This paper is structured as follows. In Section \ref{sec:background} we provide a background of the problem space, and in Section \ref{sec:problem_analysis_and_requirements} we analyse the problem and list prototype requirements. Section \ref{sec:solution} describes the solution, which we evaluate in Section \ref{sec:evaluation}. In Section \ref{sec:plan_for_future_projects} we suggest ideas for future projects. Section \ref{sec:related_work} lists related work, and we conclude on our findings in Section \ref{sec:conclusion}.
