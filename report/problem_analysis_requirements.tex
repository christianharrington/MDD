%!TEX root = ./report.tex
\section{Problem Analysis and Requirements}
\label{sec:problem_analysis_and_requirements}
In the following section we will analyse the problem, and discuss a set of requirements for a solution that is capable of editing a partial IFC model within a specific domain. This list should be thought of as a guideline for how this problem should ideally be approached.

\subsection{Problem Analysis}
\label{subsec:problem_analysis}
The separation of the construction and plumbing model makes it difficult to work on objects that are interrelated between the two models. We focus on the case of IfcFlowSegments (e.g. a pipe), IfcWallStandardCases (a regular wall) and an IfcOpeningElement (an opening in a regular wall). Jørgensen mentions that a possible solution to the problem is to allow the building service engineer to create a message with precise information about required holes, or openings, for the pipes\,\cite{jorgensen12}. This message could then be handed to the construction engineer, so that he can verify that these are properly placed. As such, the primary goal of the solution is to enable the user to work on a subset of the IFC model involving pipes, openings, and walls. In Figure \ref{fig:ifcheirachy}, a graphical representation of this subset is presented, excluding relational objects like IfcRelVoidsElement.

\begin{figure}[t]
    \centering
        \includegraphics[width=110mm]{images/IfcHeirachy.pdf}
    \caption{A graphical representation of the subset used, excluding relational objects for the sake of simplicity. The objects of the chosen subdomain are highlighted in blue.}
    \label{fig:ifcheirachy}
\end{figure}

\subsection{Requirements}
\label{subsec:requirements}
\subsubsection{Working with a Partial Model}
With the aforementioned complexities and challenges of IFC in mind, a primary focus is to be able to extract a well-specified subset of an IFC model. It is desirable to have an architecture that separates this from the rest of the workflow into one component that extracts the partial model that we are interested in. Reversely, the problem of re-inserting this partial model into the merged IFC model, should also be implemented as an encapsulated workflow component. This allows for easy reuse of the module and makes the correctness of the extraction and update processes easier to verify.

Furthermore, a clear domain definition is needed to implement and verify the extraction. Achieving this in a concise but generic way is not trivial. An evolving standard for doing this is via Model View Definitions(MVD)\,\cite{nour08}, which could allow fine-grained definitions of IFC subsets using XML. buildingSMART, a non profit organization that supports open source BIM software, propagates MVD as their standard.\footnote{buildingSMART, MVD, \url{http://buildingsmart.com/standards/mvd}} Nour discusses this and other challenges when working with partial editing on IFC\,\cite{nour08}.  Unfortunately, it was not possible to obtain the product of Nour's project. Even though MVD seems to be a promising approach for extracting IFC subsets, it is outside the scope of this project to develop the functionality ourselves. We find that simply defining a partial model with MVD is a somewhat complex task, and for the purposes of designing a single experimental DSL with only a few IFC classes, we argue that a more informal definition of the domain is sufficient, in order to simplify the development.

\subsubsection{Correct Meta Model}
It is vital for the solution that the IFC meta model is in fact correct. This point may seem obvious at first, but in our experience, finding a correct EMF meta model that reflects the actual IFC standard is difficult. In particular, finding a model with a proper serializer and deserialiser that converts from either EXPRESS or ifcXML to the corresponding Ecore instance, can be difficult. The difficulty lies in the fact, that across existing BIM software, it is not consistent how IFC models are treated, so one must be aware of inconsistencies\,\cite[p. 4]{quteprints37725}.

\subsubsection{Valid Model Transformations}
An ideal solution would feature model transformations that are verifiable and correct. By verifiable, we mean being able to trace or test that transformations actually occur in the way that we expect. By correct, we primarily mean not corrupting any model structure during a transformation.  However, with the complexity of IFC in mind, we do not require that no constraints are broken in the IFC model at the end of the transformation. As an example, we could imagine that in some IFC models all pipe objects should be referenced from some central entity. It would be difficult to require and assert that no such constraint are broken in the general case, and is a problem outside the scope of this project.

\subsubsection{A Simple DSL}
When all these technical features have been accounted for the solution still needs a simple DSL. Key to a non-experimental solution is that it displays a part of the complex source domain in a simple, manageable way.  This being a feasibility study only the inclusion of a DSL as a proof of concept, and not the syntax or usability of this, is relevant. One could imagine that a future end product would indeed be visual instead of textual.

When implemented, a simple DSL will demonstrate that the partial model editing is feasible for any subdomain of IFC. In other words the implementation will show how a small but significant target domain, like the position of a pipe and a hole in a wall, can be managed separate from the main model. Therefore, to support extensibility and reusability, it should be built with wide-spread tools like EMF and Xtext.

\subsubsection{Structural Editing}
A meaningful scope of what should be possible to do with the DSL, is the ability to edit existing values, say the position of a pipe, and more importantly, the ability to remove or add elements. The latter should be possible to resemble a real-world use case scenario, and the solution should be able to do this, of course, without corrupting the IFC instance.
\paragraph{}
This concludes the list of desirable features, but please note that it is only a core selection and that one could imagine many extensions to it. Some of these will be discussed in Section \ref{sec:plan_for_future_projects} as an idea for a future project.

