%!TEX root = ./report.tex
\section{Desirable Features}
In the following section we will discuss a set of desirable features for a solution, that is capable of editing a partial IFC model. This list should be thought of as a guideline for how this problem should ideally be approached if the project results were to be replicated. The list is formed by theoretical knowledge as well as experience gained through the course of the development of the solution.

%As stated in Section \ref{sec:building_information_modeling} (TODO double check this reference), a key challenge when working with BIM is the need for interoperability features such as being able to work on partial models separately. Even though many advanced implementations exist, most of them have mainly been focused on graphical editing for architectural domains, leaving out a world of domains that are being involved with the IFC.

\subsection{Working with a Partial model}
With the aforementioned complexities and challenges of IFC in mind, a primary focus is to be able to extract a well-specified subset of the IFC. It is desirable to have an architecture that separates this concern from the rest of the workflow into one component, that extracts the partial model that we are interested in. Reversely, the problem of re-inserting this partial model should also be implemented as a modular workflow component. This allows for easy reuse of the module and makes the correctness of the extraction process verifiable.

Furthermore, a clear domain definition is needed to implement and verify the extraction process. To do this in a concise but generic way turns out not to be entirely trivial. An evolving standard for this kind of specification is Model View Definitions(MVD)\cite{nour08}, which is a precise but extensive standard that allows fine-grained IFC subset specification in XML. However, defining a partial model with MVD is in itself a complicated task and for purposes of designing a single experimental DSL with only a few IFC classes a plain text definition of the domain is preferable to simplify the development process.\cite{mvd}

\subsection{Correct Meta Model}
When loading an IFC building instance from EXPRESS or ifcXML to Ecore it is vital for the solution that the IFC meta model is in fact correct. This point may at first seem trivial but in our experience a correct EMF meta model that reflects the actual IFC standard is difficult to obtain, especially one that comes with a proper serializer/deserialiser to Ecore. The difficulty lies in the fact that across existing implementations, it is not at all consistent how models are treated, so one must be aware of any inconsistencies in the meta model.\cite[pp. 4]{quteprints37725} The ideal solution would reuse an existing solution for this, where the IFC Wiki lists a good handful.\cite{ifcwiki}

As the IFC standard is continuously updated, it would be desirable to be able to replace the IFC meta model when it is updated. In the solution we want a well defined subset of the model, that, as a minimum, defines what is of interest in the target model, and what is it needed to make it possible to synchronize back to the source at a later stage.

\subsection{Valid Model to Model Transformation (M2M)}
An ideal solution would produce model transformations that are verifiable and correct. By verifiable, we refer to being able to trace that transformations actually occur in the way that we require. By correct, we primarily refer to not corrupting any model structure during a transformation, but also to not breaking any constraints in the domains. However, with the complexity of IFC in mind, we will relax the requirement on not breaking any constraints. As such, the ideal solution would show that a making a valid M2M is feasible, but we accept that some IFC constraints might have to be accounted for in a real-world implementation.

\subsection{A Simple DSL}
Key to the solution, is that is shows a part of the complex source domain can be represented and edited in a simple way. Targeting a simple DSL model would show this, and it will demonstrate that the target model can easily be exchanged with a new one. The latter is important, as we can imagine that future target domains, may be any arbitrary sub-domain of IFC or graphical editors. Again, we are not aiming at a rich and flexible DSL, but rather a simple language that we can easily modify. For our solution, we would like to show a target domain where we work on simple notions like the position of a pipe and a hole in a wall. To support this in being extensible and reusable, it should be build with wide-spread tools like EMF and Xtext.

\subsection{Structural Editing}
A meaningful scope of what should be possible to do with the DSL, is the ability to edit existing values, say the position of a pipe, and more importantly, the ability to remove or add elements. The latter should be possible to resemble a real-world use case scenario, and we want to be able to do this without corrupting the IFC instance.


