%!TEX root = ./report.tex
\section{Ideas for Future Projects}
\label{sec:plan_for_future_projects}
An overall goal of this pilot study was to increase the BIM knowledge of the environmentally concerned ITU research initiative, Energy Futures\footnote{For further information, see \url{http://energyfutures.itu.dk/}}. With this project as a point of departure, we have developed the following plan for future follow-up projects.

\subsection{Tools for Custom Open Source IFC Meta Model}
At the core of the prototype described in this paper is the IFC meta model. As described in Section \ref{subsec:requirements_evaluation}, our meta model generated from the ifcXML XSD has several problems, such as the lack of inverse relationships. Most of these problems could be solved by using a better, customised Ecore meta model, such as one made by Jim Steel\footnote{Jim Steel's model can be found at \url{http://www.emn.fr/z-info/atlanmod/index.php/Ecore#ifc2x3_0.1}} that we discovered late in the process. However, using this meta model would require writing serializers, which is not a trivial task. With Jim Steel's Ecore meta model as a starting point, one could a imagine an open source project in which a full-fledged serializer and deserializer for the IFC-EXPRESS language is developed, along with a revision of the meta model. This would enable future IFC developers to benefit from an independent toolset for working with IFC. These tools should support the dynamic nature of IFC, such that they are easily amended when updates to the IFC standard are released. Ideally, this Ecore meta model and the tools provided could serve as a common industry standard for working with IFC in EMF, making developers less dependent on proprietary tools.

% Study these papers:
%	"Feature-Based Survey of Model Transformation Approaches",
%	"The View Update Problem for XML",
%	"From model transformation to incremental bidirectional model synchronization" (very technical paper)
\subsection{Synchronisation}
    BIM projects should support distributed collaborative work on models. For example, we see that BIMserver is implemented with a notification system that can notify users when a model, or parts of a model, is changed. This indicates that if client software is to be useful in a real world scenario, it should be able to synchronise with the central model. As it was outside the scope of this project to require this kind of synchronisation, the final solution is rather static compared to what is desirable. Currently we update the extracted sub-set of the source model with the edits we get from PipesDSL. This is referred to as source-incrementality (\cite{czarnecki06} pp. 14) and is important for fast synchronisation with a big source model like IFC. Still we do this without considering that the source model could have changed. Also, the workflow from IFC to pipes, will always create, or overwrite, the target model without considering a scenario where an existing target model is being edited. To be able to approach the synchronisation problem, we believe it should be possible to expand on the current transformation implemented with Xtend. To be able to track changes and examine models before synchronisation, we could use traceability links to record what rules have been applied to the elements in the models. Xtend offers a class called M2MTraceElement\cite{xtendtrace} for this.



\paragraph{MVD}

\paragraph{DSL}

