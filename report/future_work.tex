%!TEX root = ./report.tex

\section{Ideas for Future Projects}
\label{sec:plan_for_future_projects}
An overall goal of this pilot study is to increase the BIM knowledge of the environmentally concerned ITU research initiative, Energy Futures. With this project as a point of departure, we have developed the following ideas for future follow-up projects.

\subsection{Tools for Custom Open Source IFC Meta Model}
At the core of the prototype, described in this paper, is the IFC meta model. As described in Section \ref{subsec:requirements_evaluation}, our meta model generated from the ifcXML XSD has several problems, such as the lack of inverse relationships. Most of these problems could be solved by using a better, customised Ecore meta model, such as one made by Steel. However, using this meta model would require writing serialisers, which is not a trivial task. With Steel's Ecore meta model as a starting point, one could imagine an open source project in which a full-fledged serialiser and deserialiser for the IFC-EXPRESS format is developed, along with a revision of the IFC meta model. This would enable developers to benefit from an independent, open source toolset for working with IFC in EMF, making them less dependent on proprietary tools.

% Study these papers:
%	"Feature-Based Survey of Model Transformation Approaches",
%	"The View Update Problem for XML",
%	"From model transformation to incremental bidirectional model synchronization" (very technical paper)
\subsection{Synchronisation}
If BIM client software is to be useful in a real world scenario when working on a partial model, it should be able to synchronise with the central model. As it was outside the scope of this project to require this kind of synchronisation for concurrent editing scenarios, our prototype is rather static compared to what is desirable for a final product. To be able to approach the synchronisation problem, we believe it is possible to expand on the current transformations implemented with Xtend. The addition of tracing would make it possible to track which transformation rules are applied to which elements in the source and target models, thus making it easier to verify the correctness of the transformations. For this purpose, Xtend offers a tracing package.\footnote{Xtend tracing package: \url{http://download.eclipse.org/modeling/m2t/xpand/javadoc/1.2/org/eclipse/xtend/util/stdlib/tracing/package-summary.html}} A project on synchronisation could investigate how a simple domain like Pipes DSL can handle proper synchronisation with a source model, possibly using BIMServer.

\subsection{Structural Changes}
Handling structural changes in the source model is challenging. With Pipes DSL, we simply insert new elements without taking changes in the source model into account, which is too na\"{i}ve to work in real-world scenarios. For example, when inserting a new pipe, an existing cost or energy object in the building model could need to reference this newly inserted object. We have allowed that such constraints can be broken by our prototype, as it is not clear how dependencies and constraints should be checked. We believe that domain specific software, capable of defining constraints for the IFC model and automatically validating it, would be useful. The project would require a clear definition of the target domain and a study of relevant real-world models.





