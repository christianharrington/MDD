%!TEX root = ./report.tex

\section{Ideas for Future Projects}
\label{sec:plan_for_future_projects}
An overall goal of this pilot study is to increase the BIM knowledge of the environmentally concerned ITU research initiative, Energy Futures. With this project as a point of departure, we have developed the following ideas for future follow-up projects.

%TODO cut down whenever Thomas has written  the 4.1 IFC meta model section
\subsection{Tools for Custom Open Source IFC Meta Model}
At the core of the prototype, described in this paper, is the IFC meta model. As described in Section \ref{subsec:requirements_evaluation}, our meta model generated from the ifcXML XSD has several problems, such as the lack of inverse relationships. Most of these problems could be solved by using a better, customised Ecore meta model, such as one made by Jim Steel that we discovered late in the process.\footnote{Jim Steel's model can be found at \url{http://www.emn.fr/z-info/atlanmod/index.php/Ecore#ifc2x3_0.1}} However, using this meta model would require writing serializers, which is not a trivial task. With Jim Steel's Ecore meta model as a starting point, one could imagine an open source project in which a full-fledged serializer and deserializer for the IFC-EXPRESS language is developed, along with a revision of the meta model. This would enable future IFC developers to benefit from an independent toolset for working with IFC in EMF. Ideally, this Ecore meta model and the tools provided could serve as an open source industry standard for working with IFC in EMF, making developers less dependent on proprietary tools.

% Study these papers:
%	"Feature-Based Survey of Model Transformation Approaches",
%	"The View Update Problem for XML",
%	"From model transformation to incremental bidirectional model synchronization" (very technical paper)
\subsection{Synchronisation}
If BIM client software, working on a partial model, is to be useful in a real world scenario, it should be able to synchronise with the central model. As it was outside the scope of this project to require this kind of synchronisation for concurrent editing scenarios, the final solution is rather static compared to what is desirable for an end product. To be able to approach the synchronisation problem, we believe it should be possible to expand on the current transformation implemented with Xtend. Tracking changes and examining models before synchronisation, one could use traceability\,\cite{czarnecki06} to record what rules have been applied to the elements in the models. Xtend offers a tracing package\footnote{Xtend tracing package: \url{http://download.eclipse.org/modeling/m2t/xpand/javadoc/1.2/org/eclipse/xtend/util/stdlib/tracing/package-summary.html}} for this. We imagine a project could investigate how a simple domain like Pipes DSL can handle synchronisation with a source model, possibly using a BIMServer.

\subsection{Structural Changes}
Handling structural changes in the source model is challenging. With Pipes DSL we simply insert new elements without taking changes in the source model into account, which is too na\"{i}ve to work in real-world scenarios. For example, when inserting a new pipe, we can imagine that a cost or energy object would need to reference this newly inserted object. Our project has allowed that we might break such constraints in the source model, as it is not clear how dependencies and constraints should be checked. We believe that domain specific software, capable of defining constraints for the IFC model and automatically validating it, would be useful. The project would require a clear definition of the target domain and a study of relevant real-world models.





